\section{Conclusão}

\par Portanto, foi possível inferir que, em um conjunto de dados desbalanceado, em que as classes com \textit{IDH} menor tem número significativamente superior de representantes. 

\par Em relação à análise geográfica, observa-se que a região Nordeste se evidencia como a mais carente em função do atributo, seguida das regões Norte e Centro-Oeste. Opostamente, as regiões Sudeste e Sul apresentaram uma proporção maior de municípios com IDH médio ou acima.

\par Na questão das correlações dos atributos preditivos, percebe-se que as quatro taxas iniciais dos dois níveis de ensino, as quais não são fortemente correlacionadas entre si, são relevantes para a taxa de distorção idade-série, a qual se configurou como atributo preditivo fortemente relacionado ao atributo alvo, além de que, as taxas que tangem à educação infantil não apresentaram tanta relevância para estimar a taxa de distorção idade-série. Além disso, nos boxplots, foi possível concluir que as taxas do par de níveis escolares, fundamental e médio, são ambas significantes para estimar a categoria de \textit{IDH} de um município, dando ênfase às taxas do ensino fundamental. Ademais, as médias de alunos e de horas-aula demonstraram ligeira relevância de predição em comparação às taxas de promoção, repetência, evasão e migração, porém a média de horas-aula possui relação mais próxima à linearidade. Além disso, viu-se que as taxas de distorção idade-série tem elevada correlação com o atributo alvo. Por fim, é razoável enfatizar que as relações ilustradas no mapa de calor foram legitimadas.